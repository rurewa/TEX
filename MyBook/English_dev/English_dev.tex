\documentclass[a4paper,16pt]{article} % Формат А4 и размер шрифта
\usepackage[pdftex,unicode, colorlinks=true, linkcolor = blue]{hyperref} % нумерование страниц, ссылки!!!!ИМЕННО В ТАКОМ ПОРЯДКЕ СО СЛЕДУЮЩИМ ПАКЕТОМ
\usepackage{cmap} % поиск в PDF
\usepackage[T2A]{fontenc} % кодировка
\usepackage[utf8]{inputenc} % кодировка исходного кода
\usepackage[english,russian]{babel} % локализаци и переносы
\usepackage{verbatim} % для длинных комментариев
\usepackage{array}
\usepackage{listings} % для вставки кода cpp
\usepackage{lscape}
%opening
\title{English-russian distonary for developer}
\date{\today} % текущая дата
\author{Александр Изотов}

\begin{document}
	%\begin{landscape}
	%\maketitle % создаёт титульный лист
	%\clearpage % новая страница
	\begin{center}
		\textbf{General}
	\end{center}
	\begin{tabular}{ | l | r |}
		\hline
			software & программное обеспечение \\ \hline
			application & приложение, программа \\ \hline
			data & данные, информация \\ \hline
			freeware & бесплатное ПО \\ \hline
			open source & ПО с открытым исходным кодом \\ \hline
			code & код (в смысле исходный код программы или в смысле код/номер ошибки или статуса, например, HTTP code — код HTTP статуса) \\
		\hline
	\end{tabular} 
	\begin{tabular}{ | l | r |}
		\hline
			software & программное обеспечение \\ \hline
			application & приложение, программа \\ \hline
			data & данные, информация \\ \hline
			freeware & бесплатное ПО \\ \hline
			open source & ПО с открытым исходным кодом \\
		\hline
	\end{tabular}
	%\end{landscape}
\end{document}